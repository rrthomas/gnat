%
% Gnat: A Simple Language
%
% Reuben Thomas
%
% 22/4-20/5/97, 7/7/02
%

% Define the semantics of the built-in functions


\documentclass[english]{article}
\usepackage{babel,a4,newlfont,amsmath,reggram,ctable}



% Alter some default parameters for general typesetting

\frenchspacing
\emergencystretch=10pt


% New commands

\newcommand{\cont}{\noindent}



\begin{document}

\title{Gnat: A Simple Language}
\author{Reuben Thomas}
\date{7th July 2002}
\maketitle



\section{Introduction}

Gnat is a new language which aims to be:

\begin{itemize}
\item simple and natural for programmers
\item easy to implement efficiently
\item convenient for all sorts of programming
\end{itemize}

It draws heavily on the functional style of Lisp, the data
structures of Lua, and the philosophy behind the design of Forth.



\section{Notation}

Gnat's grammar is described in a BNF-like notation constructed as
follows:

\begin{description}
\item[Terminal tokens]are shown \conc{thus}, which is a single token,
  not a concatenation of four.
\item[Non-terminals]are shown \nont{thus}.
\item[Concatenation]is denoted by textual concatenation.
\item[Alternatives]are separated by vertical bars, thus:
  \nont{A}\Alt\nont{B}\nont{C}. Alternation is less binding than
  concatenation, so the example just given means ``\nont{A}, or
  \nont{B} followed by \nont{C}''.
\item[Parentheses]are used to indicate binding: for example,
  \lbrac\nont{A}\alt\nont{B}\rbrac\ \nont{C} means ``\nont{A} or
  \nont{B}, followed by \nont{C}''.
\item[Optional tokens]are enclosed in square brackets, which may be
  nested:\\\conc{A cat's\opt{-tail \opt{causes}} wounds!}
\item[Zero or more]repetitions of a token are denoted by appending an
  asterisk, thus: \nont{A}\closure.
\item[One or more]repetitions of a token are denoted by appending a
  plus sign, thus: \nont{A}\pclosure.
\item[Lists]are denoted by a single terminal character before a
  repetition symbol: for example, \nont{ship}\conc{,}\pclosure\ 
  denotes a comma-separated list of ships.
\item[White space tokens]are denoted by a space within a sequence of
  concatenated tokens. The space around an alternation does not count,
  so in the following phrase there is only one white space token:
  \mbox{\nont{A} \nont{B}\Alt\nont{C}\nont{D}}. White space tokens are
  all optional (as if enclosed in square brackets).
\item[Productions]have the non-terminal being defined on the left,
  followed by an equals sign, followed by the definition:
  \mbox{\nont{insect} = \nont{head}\nont{thorax}\nont{abdomen}}.
\end{description}



\section{The core language}

\subsection{Tokens}

A program consists of a sequence of tokens; their grammar is given by:

\begin{bnfc}
\item[tok]\nont{res}\Alt\nont{id}\Alt\nont{char}
\item[id]\nont{alpha}\lbrac\nont{letter}\alt\nont{digit}\rbrac\closure\Alt
  \nont{symbol}\pclosure
\item[symbol]!\Alt \#\Alt \$\Alt \%\Alt \&\Alt '\Alt *\Alt +\Alt -\Alt .%
\Alt /\Alt :\Alt<\Alt =\Alt >\Alt ?\Alt @\Alt \verb|\|\Alt \verb|^|\Alt%
\_\Alt`\Alt|\Alt\verb|~|
\item[digit]\textrm{A decimal digit from 0--9}
\item[alpha]\textrm{An upper or lower-case letter from A--Z}
\item[letter]\nont{alpha}\Alt\_
\item[char]\textrm{Any character}
\end{bnfc}

Any number of spaces, linefeeds (ASCII 10) and tabs (ASCII 9), form a
white space token. A comment, which starts with \conc{//} and ends
with \verb|\\|, also counts as white space. Comments may be nested.

A program is formed from a string of 8-bit ASCII characters as
follows:

\begin{enumerate}
\item Find the longest prefix of the string that is a token, and
  delete it from the string.
\item If the token is a white space token then discard it.
\item If the token is a \nont{strings}, concatenate all the
  \nont{string}s which form it together to form one large
  \nont{string}.
\item Append the token to the program.
\item Repeat from step 1 until no characters remain in the string.
\end{enumerate}

\subsubsection{Reserved identifiers}

The identifiers given by \nont{res} are reserved, and may not be used
as \nont{var}s.

\begin{bnfc}
\item[res]\#\Alt\$\Alt\$\$\Alt\$=\Alt.\Alt:\Alt:=\Alt=>\Alt|=\Alt null\Alt
false\Alt true\Alt if\Alt then\Alt else \Alt while\Alt do
\end{bnfc}


\subsection{Sequence}

\begin{bnfc}
\item[seq]\nont{exp} ; \pclosure
\end{bnfc}

\cont A sequence of expressions (\nont{seq}) contains expressions
separated by semi-colons; since an expression may be empty, a sequence
may also end in a semi-colon. To evaluate a \nont{seq} the expressions
are evaluated from left to right. The value of a sequence is the value
of the last expression.


\subsection{Expression}

\begin{bnfc}
\item[exp]\opt{\nont{hand}}\Alt\nont{func}\Alt\nont{app}
\end{bnfc}

\cont An expression (\nont{exp}) is a handled expression
(\nont{hand}), a function (\nont{func}), an application (\nont{app})
or the empty expression, whose value is \conc{null}.


\subsection{Exception}

\begin{bnfc}
\item[hand]\nont{exp} catch \nont{catches}
\item[catches]\nont{catch}\Alt\nont{catches} | \nont{catch}
\item[catch]\nont{exc} => \nont{exp}
\item[exc]\nont{id}
\end{bnfc}

\cont


\subsection{Application}

\begin{bnfc}
\item[app]\nont{val}\pclosure
\end{bnfc}

\cont An application (\nont{app}) is a list of values. The value of an
application $av$ is the value obtained by applying the value of $a$ to
the value of $v$. The values of these various applications are
described below in the appropriate places. Any application not
described below has the value \conc{null}.


\subsection{Function}

\begin{bnfc}
\item[func]\nont{var} => \nont{exp}
\end{bnfc}

\cont The value of \conc{($x$ => $e$)$v$} is the value of $e$ after
the value of $v$ has been substituted for all free occurrences of $x$.
Within $e$ the identifier \conc{<=} has the value \mbox{\conc{$x$ =>
    $e$}}. The value of a function is itself.


\subsection{Value}

\begin{bnfc}
\item[val]\nont{list}\Alt\nont{scope}\Alt null
\end{bnfc}

\cont A value (\nont{val}) is either a list (\nont{list}), a scope
(\nont{scope}) or \conc{null}. The value of \conc{null} is itself.
\conc{null} is the only value that is not a list.


\subsection{Slice}

There are two slice functions: the small slice, \conc{\$}, and the big
slice, \conc{\$\$}. The value of the small slice \conc{\$$nl$} is that
of the $n$th element of $l$. If $|n|$ is outside the range
\conc{$1$\dots\#$l$} then the value of the slice is \conc{null}. The
value of the big slice \conc{\$\$$l_1$$l_2$} is a list of elements of
$l_2$, according to the value of the elements of $l_1$. $l_1$ may
contain three sorts of elements:

\begin{description}
\item[$a$]An atom whose value is $n$ selects \conc{$l_2$\$$n$}.
\item[\conc{[$a_1$]}]A list with one element whose value is $n$
  selects \conc{$l_2$\$$\operatorname{sgn} n\ldots l_2$\$$n$}.
\item[\conc{[$a_1$,$a_2$]}]A list with two elements whose values are
  $n_1$ and $n_2$ selects \conc{$l_2$\$$n_1\ldots l_2$\$$n_2$}. If the
  $n_1$th element comes after the $n_2$th, no elements are selected.
\end{description}


\subsection{Selection}

The value of a selection \conc{|$vl$} is the index of the first
element from the left in $l$ whose first element is $v$. If there is
no such element in $l$ then the value of the selection is \conc{null}.


\subsection{List}

\begin{bnfc}
\item[list]{[} \nont{app},\pclosure\ ]\Alt\\{[} \nont{elem},\pclosure\ ]\Alt\\\nont{atom}
\item[elem]\nont{ord}\Alt\nont{key}
\item[ord]\nont{app} |= \nont{app}
\item[key]\nont{app} \$= \nont{app}
\end{bnfc}

\cont Lists may be atomic (\nont{atom}) or non-atomic. A list's
elements are implicitly numbered, and may also be indexed
associatively. An atomic list is a bit string.

The elements of a list are numbered left to right consecutively from
one upwards, and right to left from minus one downwards.

Lists may be applied to one another. The value of the application
$l_1l_2$ is the result of successively modifying $l_1$ with each
element of $l_2$, from left to right. The results of the various
possible modifications of a list $l$ by a modifier are as follows:

\begin{ctabular}{cp{3in}}
  %\tabstrut
  \bf Modifier & \bf Result \\ \hline \nont{app}: $a$ &
  An extra element whose value is $a$ is added to the right-hand
  end of $l$, unless the value of $a$ is \conc{null}, in which
  case it is ignored. \\
  \nont{ord}: \conc{$a_1$ |= $a_2$} &
  The value of element $a_1$ of $l$ is set to the value of $a_2$;
  if $|a_1|$ is outside the range \conc{$1$\dots\#$l$} then the
  \nont{ord} has no effect. \\
  \nont{key}: \conc{$a_1$ \$= $a_2$} &
  The first element of $l$ from the left whose first element has
  the same value as $a_1$ is replaced by the value of
  \conc{[$a_1$][$a_2$]}; if no such element exists then the new
  element is added to the right-hand end of the list. If $a_1$ has
  the value \conc{null} then the \nont{key} has no effect. \\
\end{ctabular}

\cont The value of a list $l$ is the value of \conc{[]$l$}. The value
of \conc{$l$\,null} is the value of $l$, and the value of
\conc{null\,$l$} is \conc{null}.


\subsection{Scope}

\begin{bnfc}
\item[scope]\lbraceconc\ \nont{ass} , \closure\ \rbraceconc
\item[ass]\nont{var} := \nont{exp}
\end{bnfc}

\cont A scope is an abstract list, with elements that are labelled
rather than numbered. In a scope \conc{\lbraceconc $i_1$ := $e_1$ ,
  \dots\ , $i_n$ := $e_n$ \rbraceconc} the expressions $e_1$ to $e_n$
are evaluated, and the results assigned to the variables $i_1$ to
$i_n$. If any of $i_1$ to $i_n$ are evaluated in the course of
evaluating $e_1$ to $e_n$, their value will be \conc{null}; however,
any unevaluated occurrences will refer to the variable defined in the
scope. \emph{This means that \conc{\lbraceconc a := 2, b :=
    a+1\rbraceconc} will set \conc{a} to 2 and \conc{b} to
  \conc{null}, but recursive function definitions will work as
  expected. This might change if a different evaluation strategy is
  used, e.g. call-by-need reduction might allow it to work, since
  neither expression would be evaluated until later.}

If a variable is assigned to more than once in a scope, its value is
that of the last expression assigned to it.

Scopes may be applied to one another. If $s_1$ is
\conc{\lbraceconc$a_1$,\dots,$a_n$\rbraceconc} and $s_2$ is
\conc{\lbraceconc$b_1$,\dots,$b_m$\rbraceconc} then the value of
$s_1s_2$ is
\conc{\lbraceconc$a_1$,\dots,$a_n$,$b_1$,\dots,$b_m$\rbraceconc}.


\subsection{Version}

\begin{bnfc}
\item[ver]\nont{scope} \nont{val}
\end{bnfc}

The value of a version \conc{$sv$} is the value of $v$ when free
occurrences of the elements of $s$ in $v$ are given the values
assigned to them in $s$. \emph{It's not clear where a version may occur
  in the syntax.}


\subsection{Atom}

\begin{bnfc}
\item[atom]\#\nont{list}\Alt\nont{var}\Alt false\Alt true\Alt(\nont{seq})
\item[var]\nont{id}
\end{bnfc}

\cont An atom is either a count (\conc{\#$l$}), a variable
(\nont{var}) the single bit \conc{false} (0) or \conc{true} (1), or a
bracketed sequence of expressions (\conc{(\nont{seq})}.

The value of a \nont{var} to which no value has been assigned is
\conc{null}. The value of a \nont{count} \conc{\#$a$} is the number of
elements in $a$.



\end{document}
